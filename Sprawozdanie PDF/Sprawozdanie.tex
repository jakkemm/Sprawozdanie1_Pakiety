\documentclass{article}\usepackage[]{graphicx}\usepackage[]{xcolor}
% maxwidth is the original width if it is less than linewidth
% otherwise use linewidth (to make sure the graphics do not exceed the margin)
\makeatletter
\def\maxwidth{ %
  \ifdim\Gin@nat@width>\linewidth
    \linewidth
  \else
    \Gin@nat@width
  \fi
}
\makeatother

\definecolor{fgcolor}{rgb}{0.345, 0.345, 0.345}
\newcommand{\hlnum}[1]{\textcolor[rgb]{0.686,0.059,0.569}{#1}}%
\newcommand{\hlstr}[1]{\textcolor[rgb]{0.192,0.494,0.8}{#1}}%
\newcommand{\hlcom}[1]{\textcolor[rgb]{0.678,0.584,0.686}{\textit{#1}}}%
\newcommand{\hlopt}[1]{\textcolor[rgb]{0,0,0}{#1}}%
\newcommand{\hlstd}[1]{\textcolor[rgb]{0.345,0.345,0.345}{#1}}%
\newcommand{\hlkwa}[1]{\textcolor[rgb]{0.161,0.373,0.58}{\textbf{#1}}}%
\newcommand{\hlkwb}[1]{\textcolor[rgb]{0.69,0.353,0.396}{#1}}%
\newcommand{\hlkwc}[1]{\textcolor[rgb]{0.333,0.667,0.333}{#1}}%
\newcommand{\hlkwd}[1]{\textcolor[rgb]{0.737,0.353,0.396}{\textbf{#1}}}%
\let\hlipl\hlkwb

\usepackage{framed}
\makeatletter
\newenvironment{kframe}{%
 \def\at@end@of@kframe{}%
 \ifinner\ifhmode%
  \def\at@end@of@kframe{\end{minipage}}%
  \begin{minipage}{\columnwidth}%
 \fi\fi%
 \def\FrameCommand##1{\hskip\@totalleftmargin \hskip-\fboxsep
 \colorbox{shadecolor}{##1}\hskip-\fboxsep
     % There is no \\@totalrightmargin, so:
     \hskip-\linewidth \hskip-\@totalleftmargin \hskip\columnwidth}%
 \MakeFramed {\advance\hsize-\width
   \@totalleftmargin\z@ \linewidth\hsize
   \@setminipage}}%
 {\par\unskip\endMakeFramed%
 \at@end@of@kframe}
\makeatother

\definecolor{shadecolor}{rgb}{.97, .97, .97}
\definecolor{messagecolor}{rgb}{0, 0, 0}
\definecolor{warningcolor}{rgb}{1, 0, 1}
\definecolor{errorcolor}{rgb}{1, 0, 0}
\newenvironment{knitrout}{}{} % an empty environment to be redefined in TeX

\usepackage{alltt}
\usepackage[utf8]{inputenc}
\usepackage{polski}
\usepackage{tgpagella}
\usepackage{hyperref}
\author{Emil Olszewski, Jakub Kempa}
\date{\today}
\title{Sprawozdanie 1}
\IfFileExists{upquote.sty}{\usepackage{upquote}}{}
\begin{document}
\maketitle

% ------------------- STRESZCZENIE --------------------------
\begin{abstract}

Przedmiotem analizy są dane ze zbioru zawierającego informacje na temat 
męskich trójboistów zrzeszonych w ramach federacji IPF. \href{https://gitlab.com/openpowerlifting/opl-data}{Dane} zostały udostępnione na warunkach licencji GNU AGPLv3. 
Głównymi zmiennymi, które będą nas interesować będą \textit{Age}, \textit{Sex} (zmienne kategoryczne określające wiek i płeć zawodnika) oraz zmienne ciągłe \textit{BodyweightKg} oraz \textit{TotalKg}, które wyrażają odpowiednio masę ciała zawodnika oraz wynik całkowity, będący sumą wyników w trzech kategoriach: \textit{wyciskanie na ławce, przysiad ze sztangą, martwy ciąg}.

Do analizy danych użyto języka \textit{R}.

\end{abstract}

\section{Opis danych} 
Pod uwagę wzięto tylko zawodników płci męskiej, dla których dostępny był pełen zestaw danych dotyczący wyników uzyskanych w każdym z trzech bojów. Ograniczono się dodatkowo do cenzusu wiekowego w przedziale od 16 do 40 lat oraz rozpatrywano tylko wyniki uzyskane w kategorii RAW (kategoria, która zabrania używania sprzętu dającego przewagę mechaniczną np. koszulek do wyciskania, kaftanów itd. Jest to klasyczna kategoria trójboju siłowego). \\
Skoncentrowano się na sześciu kluczowych zmiennych:

    \begin{itemize}
        \item \textbf{Age (wiek zawodnika)}: Ta zmienna kategoryczna reprezentuje wiek zawodnika zawodnika. Choć intuicynie może się wydawać, że jest ona bardzo istotna (wraz z wiekiem witalność sportowa powinna spadać), tak jednak w weightliftingu wiek nie jest kluczowy. Wykażą to późniejsze analizy w sprawozdaniu.
        \item \textbf{Sex (płeć zawodnika)}: Płeć jako zmienna kategoryczna naturalnie istotnie wpływa na wyniki. Kobiety nie będą osiągać tak samo wysokich wyników, w tych samych kategoriach, co mężczyźni. Nasza analiza będzie skupiać się jedynie na mężczyznach.
        \item \textbf{BodyweightKg (masa ciała zawodnika)}: Ta zmienna ciągła reprezentuje masę ciała zawodnika w kilogramach. Masa ciała jest istotnym parametrem w trójboju siłowym, ponieważ klasyfikuje zawodników w odpowiednie kategorie wagowe i może wpływać na ich wydajność w zawodach.
        \item \textbf{TotalKg (Całkowity wynik)}: Jako zmienna ciągła, całkowity wynik odnosi się do sumy maksymalnych ciężarów, które zawodnik podniósł w trzech próbach: przysiadzie, wyciskaniu leżąc i martwym ciągu. Jest to główny wskaźnik wydajności w trójboju siłowym, odzwierciedlający siłę i umiejętności zawodnika. W dalszej części raportu będziemy używać określeń takich jak \textbf{Wynik sumaryczny}, \textbf{Wynik total} czy po prostu \textbf{total}.
    \end{itemize}
     
Wiersze, w których w naszych kategoriach pojawiają się braki danych usuwamy. Postanowiliśmy z otrzymanych danych wydzielić losową próbę o długości 1500. Poniżej zamieszczone są informacje o naszych danych.

\begin{knitrout}
\definecolor{shadecolor}{rgb}{0.969, 0.969, 0.969}\color{fgcolor}\begin{kframe}


{\ttfamily\noindent\itshape\color{messagecolor}{\#\# -- Attaching core tidyverse packages ------------------------ tidyverse 2.0.0 --\\\#\# v dplyr \ \ \ \ 1.1.3 \ \ \ \ v readr \ \ \ \ 2.1.4\\\#\# v forcats \ \ 1.0.0 \ \ \ \ v stringr \ \ 1.5.0\\\#\# v ggplot2 \ \ 3.4.3 \ \ \ \ v tibble \ \ \ 3.2.1\\\#\# v lubridate 1.9.3 \ \ \ \ v tidyr \ \ \ \ 1.3.0\\\#\# v purrr \ \ \ \ 1.0.2 \ \ \ \ \\\#\# -- Conflicts ------------------------------------------ tidyverse\_conflicts() --\\\#\# x dplyr::filter() masks stats::filter()\\\#\# x dplyr::lag() \ \ \ masks stats::lag()\\\#\# i Use the conflicted package (<http://conflicted.r-lib.org/>) to force all conflicts to become errors\\\#\# \\\#\# Attaching package: 'dbplyr'\\\#\# \\\#\# \\\#\# The following objects are masked from 'package:dplyr':\\\#\# \\\#\# \ \ \ \ ident, sql}}\end{kframe}
\end{knitrout}
\begin{knitrout}
\definecolor{shadecolor}{rgb}{0.969, 0.969, 0.969}\color{fgcolor}\begin{kframe}
\begin{verbatim}
##       Age         BodyweightKg       TotalKg     
##  Min.   : 8.00   Min.   : 29.85   Min.   : 97.5  
##  1st Qu.:19.00   1st Qu.: 74.20   1st Qu.:440.0  
##  Median :23.00   Median : 85.62   Median :510.0  
##  Mean   :25.61   Mean   : 87.85   Mean   :511.5  
##  3rd Qu.:30.00   3rd Qu.: 98.00   3rd Qu.:587.5  
##  Max.   :72.00   Max.   :196.10   Max.   :925.0
\end{verbatim}
\end{kframe}
\end{knitrout}



\end{document}
